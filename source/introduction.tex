\section{Introduction}
\label{sec:introduction}

In the early 1950s we unlocked two nuclei: fusion and the cell.
It is remarkable how little these discoveries have changed our lives.
We still resemble our ancestors from thousands of years ago.
With continual advances in technology, we now stand on the verge of certain very dramatic changes.
The biotech and genetic revolutions are poised to fundamentally change our lives.

Should we be playful, excited or terrified by this?
We have been capable of genetic manipulation for quite some time.
For thousands of years, we have been breeding plants and animals to fulfill our growing needs.
Recently, our tools have become orders of magnitude more powerful, more precise, and less expensive in time and materials.
Where once we had to rely on chance for desirable traits, we can now be extremely precise with the changes we want.

Advances in biotechnology is rapidly opening up the world of human genetic modification.
Once we start changing our traits, we will not stop.
We cannot think of nature as separate from us and of ourselves as ``screwing with nature.''
We are nature, nature is us and nature is always changing.
This makes nature natural to hack and we Homo sapiens, with tool usage as our comparative advantage, are natural hackers.

Today we expect that our next phone will be better and faster than the last, but we have this adherence to our biology.
``I am Homo sapiens and my kids are Homo sapiens.''
We are rapidly moving into a world where we will be able to edit our genomes at will.
CRISPR is among the most prominent and powerful technologies at our disposal and lets us make arbitrarily precise edits to various genomes.
Every day there are not just additiona; stories of applications of CRISPR, but new types of gene-editing tools.
The idea that we can rewrite our biology is going to cause a fundamental shift in the way we see ourselves.

Computational biology and data science are driving the expansion of the depth and breadth of our knowledge of ourselves.
As biologists and data scientists at the heart of these revolutions, we bear the lion's share of the responsibility in moving the world forward.
With growing insight into ourselves, we are starting to see some profoundly disturbing data, along with unsavory interpretations of those data.
Many of our cherished values as human beings are coming into conflict with the emerging science.
There is often no good position to hold on these issues and we struggle to communicate this to the world.
With this glut of information about ourselves, our technological and social needs are at risk of clashing in a profoundly destructive manner.
This is where we find ourselves blocked in terms of the theme of this essay.
Can we use our new powers to find a graceful exit from these very powerful conundrums?
Conversation might be the only portal available.

Our story will play out along three trajectories: the utopian, the dystopian, and the impasse.
Only through honest and rigorous conversation and by engaging with the public from the very beginning can we avoid stagnation and impasse.
Through such conversation, we will each be able to move the dial a little in the direction of the utopian or a little in the direction of the dystopian.
We need to infuse the conversations about values and norms into the development of these powerful technologies while we also integrate science into our social decisions.
