\section{Introduction}
\label{sec:introduction}

In the early 1950s we unlocked two nuclei: fusion and the cell.
It is remarkable how little these discoveries have changed our lives and how much we still resemble our ancestors from thousands of years ago.
BDue to continual advances in these technologies, we stand on the precipice of great change.
Should we be playful, excited or terrified with these new tools?

We have been capable of genetic manipulation for quite some time.
For thousands of years, we have been breeding plants and animals to meet our growing needs.
We relied on chance for desirable traits to emerge.
Recently, our tools have become orders of magnitude more powerful, more precise, and less expensive.
Advances in biotechnology allow us to make precise changes, and once we start changing our traits, we will not stop.
We cannot think of nature as separate from us and of ourselves as ``screwing with nature.''
We are nature, nature is us and nature is always changing.
This makes nature natural to hack and we Homo sapiens, with tool usage as our comparative advantage, are natural hackers.

We expect that our next phone will be better and faster than the last, but we think of our biology as static.
``I am Homo sapiens and my kids are Homo sapiens.''
With IVF, we can now take conception outside of the human body and with technologies like CRISPR, we will be able to edit our genomes at will.
Every day there are not just additional stories of applications of CRISPR, but new types of gene-editing tools.
This ability to rewrite our source-code is going to fundamentally shift our self-perception.
As biologists and data scientists spearheading this revolution, we bear the lion's share of responsibility for moving the world forward.

Our story will play out along three trajectories: the utopian, the dystopian, and the stagnant.
Only through honest and rigorous conversation and by engaging with the public from the very beginning can we avoid stagnation.
Through such conversation, we will each be able to move the dial a little in the direction of utopian or a little in the direction of dystopian.
We need to infuse conversations about values and norms into the development of these technologies while also integrating science into our social decisions.

As our understanding of ourselves deepens, we begin to encounter data that challenges many of our cherished values.
We must continue with the science risk being left behind.
This advancement in science also threatens to shatter the ethical codes we have developed and fought for over thousands of years.
With this upcoming clash between our technological needs and our social needs, can we use our newfound powers to find a resolution?
Is there a portal that will let us make a graceful exit from this conundrum?
