\section{Introduction}
\label{sec:introduction}

In the early 1950s we unlocked two nuclei: fusion and the cell.
It is remarkable how little these discoveries have changed our lives and how much we still resemble our ancestors from thousands of years ago.
With continual advances in these technologies, we stand on the precipice of great change.
Should we be playful, excited or terrified by this?

We have been capable of genetic manipulation for quite some time.
For thousands of years, we have been breeding plants and animals to meet our growing needs.
We had to rely on chance for desirable traits to emerge.
Recently, our tools have become orders of magnitude more powerful, more precise, and less expensive in time and materials.
Advances in biotechnology allow us to precisely aim towards the changes that we want.
Once we start changing our traits, we will not stop.
We cannot think of nature as separate from us and of ourselves as ``screwing with nature.''
We are nature, nature is us and nature is always changing.
This makes nature natural to hack and we Homo sapiens, with tool usage as our comparative advantage, are natural hackers.

We expect that our next phone will be better and faster than the last, but we think of our biology as static.
``I am Homo sapiens and my kids are Homo sapiens.''
With technologies like CRISPR, we are rapidly moving into a world where we will be able to edit our genomes at will.
Every day there are not just additional stories of applications of CRISPR, but new types of gene-editing tools.
The ability to rewrite our source-code is going to cause a fundamental shift in our self-perception.
As biologists and data scientists spearheading this revolution, we bear the lion's share of responsibility in moving the world forward.

Our story will play out along three trajectories: the utopian, the dystopian, and the impasse.
Only through honest and rigorous conversation and by engaging with the public from the very beginning can we avoid stagnation and impasse.
Through such conversation, we will each be able to move the dial a little in the direction of the utopian or a little in the direction of the dystopian.
We need to infuse the conversations about values and norms into the development of these technologies while also integrating science into our social decisions.
With this upcoming clash between our cherished values and the emerging data, can we use our new powers to find a graceful exit?
