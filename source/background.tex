\section{Background}
\label{sec:background}

IVF. Embryo selection.
Genetically edited babies in china.
KRYSPR.

Selective breeding of plants and animals has been practiced since prehistoric times \cite{Buffum1909}.
Species of wheat, rice, and dogs are different from their wild ancestors.
Maize, i.e. corn, required especially large changes from teosinte, its wild form, and was selectively bred in Mesoamerica.
Sources as much as 2,000 years old give advice on selecting animals for different purposes, and these works cite still older authorities, such as Mago the Carthaginian \cite{Lush2017}.
These old methods are clumsy and imprecise.
We would breed species together in the hopes of generating some desired phenotypic trait.
Some of our other methods were quite outrageous.
We would bombard plants with radiation, find the rare plant with a desirable trait, e.g. seedless grapes, and then produce more of that plant.

With the invention of techniques around Recombinant DNA, we developed Genetically Modified Organisms, i.e. GMOs.
These include Golden Rice \cite{Xudong2000}, herbicide \cite{Funke2006} and insect \cite{Paine2005} resistant crops.
Recognizing the power of Recombinant DNA, the scientists involved congregated at Asilomar in the early 1970s to establish principles to ensure the safety of this budding technology \cite{Berg1975}.
However, there was no public engagement from the start and so there is still much fear among the public.
Thus, the science has not realized its potential.

With animals, we have bred swine for meat \cite{OSU2011}, horses for performance \cite{Evans2000}, and pets such as dogs, cats, and birds.
In 2013, researchers at the University of Istanbul created glow-in-the-dark rabbits \cite{Rojhan2013}.
Sea jelly genes were engineered into the bunnies' genomes, granting them luminescence.
These bunnies were bred in an effort to create animals capable of producing medicines in their milk.
We have continued such genetic experimentation on animals, with mice being the chief test subjects.

Our first steps down the path of human genetic engineering have been In Vitro Fertilization.
IVF involves the extraction of human gametes after which a small number of fertilized embryos are created in a lab.
After some time, these are implanted in a womb with the hope that some will take root and a child will be borne to term.
The first IVF pregnancy was reported in 1973 \cite{Kretzer1973} but it resulted in an early miscarriage.
The first IVF birth occurred in Oldham, England on July 25, 1978 \cite{Steptoe1978}.
Since then IVF has grown in popularity.
IVF births account for 2\% of the births in the United States, 5\% in Japan, and 10\% in Norway and Denmark \cite{Metzl2019}.

With the completion of the Human Genome Project and with the cost of sequencing trending towards negligibility \cite{Wetterstrand2019}, we can now read many individual genomes.
