\section{Background}
\label{sec:background}

The history of genetic manipulation by humans includes plant and animal breeding, GMOs, IVF, the Human Genome Project, and CRISPR.

\subsection{Selective Breeding}

Selective breeding of plants and animals has been practiced since prehistoric times \cite{Buffum1909}.
Modern species of wheat, rice, and dogs are different from their wild ancestors.
Maize, i.e. corn, required especially large changes from teosinte, its wild form, and was selectively bred in Mesoamerica.
Sources from as long as 2,000 years ago give advice on selecting animals for different purposes, and these works cite still older authorities, such as Mago the Carthaginian \cite{Lush2017}.
These old methods are clumsy and imprecise.
We would breed species together in the hopes of generating some desired phenotypic trait.
Some of our other methods were quite outrageous.
We would bombard plants with radiation, find the rare plant with a desirable trait, e.g. seedless grapes, and then produce more of that plant.

With animals, we have bred swine for meat \cite{OSU2011}, horses for performance \cite{Evans2000}, and dogs, cats, and birds as pets.
In 2013, researchers at the University of Istanbul created glow-in-the-dark rabbits \cite{Rojhan2013}.
Sea jelly genes were engineered into the bunnies' genomes, granting them luminescence.
These bunnies were bred in an effort to create animals capable of producing medicines in their milk.
We have continued such genetic experimentation on animals, with mice being the chief test subjects.

\subsection{Genetically Modified Organisms}

With the invention of techniques around Recombinant DNA, we developed Genetically Modified Organisms, i.e. GMOs.
These include Golden Rice \cite{Xudong2000}, herbicide-resistant \cite{Funke2006} and insect-resistant \cite{Paine2005} crops.
Recognizing the power of Recombinant DNA, the scientists involved congregated at Asilomar in the early 1970s to establish principles to ensure the safety of this budding technology \cite{Berg1975}.
However, there was no public engagement during the early developments and fear among the public has persisted since.
Thus, the science has not realized its potential.

\subsection{In Vitro Fertilization}

Our first steps down the path of human genetic engineering have been with IVF.
This involves the extraction of human gametes after which a small number of fertilized embryos are created in a lab.
After some time, these are implanted in a womb with the hope that some will take root and a child will be borne to term.
The first IVF pregnancy was reported in 1973 \cite{Kretzer1973} but it resulted in an early miscarriage.
The first IVF birth occurred in Oldham, England on July 25, 1978 \cite{Steptoe1978}.
Since then, IVF has grown in popularity.
IVF births account for 2\% of the births in the United States, 5\% in Japan, and 10\% in Norway and Denmark \cite{Metzl2019}.
These numbers will continue to rise.
Once we take conception outside the human body, entirely new sets of possibilities will lay open for us as a species.

\subsection{Genome Sequencing}

Sice the completion of the Human Genome Project, the cost of sequencing has been falling towards negligibility \cite{Wetterstrand2019}.
With this ability to cheaply read our genomes, we are starting to demystify our biology.
Among other developments, we have learned how to identify Down Syndrome during pre-natal screening \cite{Natoli2012}, and are able to diagnose various genetic disorders in fertilized embryos during IVF \cite{Rycke2017}.
We have also been able to identify and categorize over 5,000 single-gene-mutations, i.e. Mendelian disorders.
These include sickle cell anaemia, color blindness, muscular dystrophy, and cystic fibrosis.
This has opened up choices for which embryos to implant through IVF.

\subsection{CRISPR}

In addition to reading genomes, we now have the ability to arbitrarily edit genomes.
The inventions of CRISPR, Clustered Regularly Interspaced Short Palindromic Repeats, and Cas9, CRISPR-associated protein 9, have opened up the world of genetic editing \cite{Zhang2014}.
CRISPR works on genomes much like a text editor works on strings.
There is a guide-RNA, the cursor, that can be placed at any desired nucleotide.
From here, string of any length can be deleted or replaced by a different string.
This lets us make extremely precise edits to genomes at low expense of power and time.
The near-term applications of tools like CRISPR include potentially curing all Mendelian disorders, while long-term applications include rewriting large sections of human genomes to make ``designer babies.''
In fact, the first genetically edited babies were born in China in October 2018 \cite{crisprbabies2018}.
Dr. He Jiankui announced the birth in videos on YouTube saying that he edited the twin girls' CCR5 gene to grant them increased resistance against HIV.
The Russian scientist Dennis Rebrikov has announced that five of his patients have signed up for gene-edited babies \cite{Cohen2019}.
