\section{Viewpoints}
\label{sec:viewpoints}

national level cultures around this topic.

The emerging biotechnologies are going to have species-wide impacts.
It is to be expected that people have many different views around various developments.

\subsection{Public Engagement and Discourse}

It is essential to keep the public engaged, from the very beginning, with new and powerful technologies and their implications for our society.
While the scientists involved with Recombinant DNA showed a great deal of responsibility with their conference at Asilomar, they lacked this crucial engagement with the public.
Scientists often communicate in a coded language.
While this is great for idea transmission within academic communities, the density makes it opaque for the general public to approach.
The public lack of understanding of this field has bred fear and uncertainty around GMOs.
The ensuing lack of support and engagement from the public has stunted growth in this field.
Indeed, dissociation from the public discourse around a science can kill that science.

However, there is a strong argument to be made for the position that we should let these developments fly under the radar.
Scientists are, by and large, responsible and we should let them do their work.
The near-term application of technologies like genomic sequencing of embryos during IVF include diagnoses and cures for such terrible ailments as Tay-Sachs disease.
Poking the hornets' nest may cause people on one side or the other of the abortion issue to build and man new barricades.
We should let this emerge in much the same way as IVF did where, by the time the issue drew public attention, people in evangelical communities were already speaking in their churches of the miracle of life.

\subsection{Designer Babies}

We have many bugs in our genetic code.
These cause our children to die from genetic diseases and our elderly to suffer from the likes of dementia and Alzheimer's.
Having already identified thousands of genetic disorders, we know what to look for when sequencing the embryos during IVF.
We will know, for example, if a child, if carried to term, will die of Tay-Sachs.
Data from European countries show that couples that receive a diagnosis of Down Syndrome from a pre-natal screening at three months, up to 93\% choose to abort \cite{Natoli2012}.
From this we can impute that almost no one will choose to implant an embryo that we know would be a child who will suffer and die from a genetic disorder.

China's biophysicist He Jiankui is responsible for teh birth of the first known genetically-edited babies in the world.
His "achievement" was initially lauded in China in the People's Daily as "a triumph of Chinese science".
(The article has since been removed from the online archive of the People's Daily.)
There was swift international condemnation of his actions \cite{crisprbabies2018}.
The event was controversial for a number of reasons.
The consent of the parents was misinformed.
He Jiankui did not obtain approval from the hospital.
The father had HIV while the mother did not.
In such a circumstance in a country with advanced medicine such as China, or the United States, there are many ways for the couple to have a child who will not inherit HIV.
He Jiankui targeted the CCR5 gene to make it similar to a mutation that some europeans have.
The mutation is two disrupted copies of the gene that grant the carrier increased resistance to HIV but makes them more susceptible to the West Nile virus.
He Jiankui was not trying to solve an existing problem but was trying to make an enhancement.
A few months later, a report came out that mice with the same CCR5 mutation were faster at maze-running \cite{Regalado2019}.
This sparked rumors that the babies were engineered to be smarter.
Soon after, a study was released, analyzing data of almost half a million people from the UK Biobank, that found a link between people with the same CCR5 mutation and their lifespans \cite{Wei2019}.
People with the mutations were living shorter lives.
