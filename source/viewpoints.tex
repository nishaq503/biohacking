\section{Viewpoints and Personal Opinions}
\label{sec:viewpoints}

The emerging biotechnologies are going to have species-wide impacts.
It is to be expected that people have many different views around various developments.

\subsection{Public Engagement and Discourse}

It is essential to keep the public engaged, from the very beginning, with new and powerful technologies and their implications for our society.
While the scientists involved with Recombinant DNA showed a great deal of responsibility with their conference at Asilomar, they lacked this crucial engagement with the public.
Scientists often communicate in a coded language.
While this is great for idea transmission within academic communities, the density makes it opaque for the general public to approach.
The public lack of understanding of this field has bred fear and uncertainty around GMOs.
The ensuing lack of support and engagement from the public has stunted growth in this field.
Indeed, dissociation from the public discourse around a science can kill that science.

However, there is a strong argument to be made for the position that we should let these developments fly under the radar.
Scientists are, by and large, responsible and we should let them do their work.
The near-term application of technologies like genomic sequencing of embryos during IVF include diagnoses and cures for such terrible ailments as Tay-Sachs disease.
Poking the hornets' nest may cause people on one side or the other of the abortion issue to build and man new barricades.
We should let this emerge in much the same way as IVF did where, by the time the issue drew public attention, people in evangelical communities were already speaking in their churches of the miracle of life.

I believe that we are talking about technologies that are everyone's business and we need to respect each other enough to have this conversation out in the open.
We cannot afford to make the same mistake that we did in the beginning of the GMO era.
Will our conversation around genetic manipulation lead us to adaptive fears which would cause us to come up with wise restrictions for how we do our science and engineering?
Or will we spend our time worrying about the wrong thing because some benign modification, because of the way that it is phrased, catches the public eye and causes a panic that shuts down the science?
Can we figure out which fears are adaptive and which are maladaptive?

\subsection{Designer Babies}

We have many bugs in our genetic code.
These cause our children to die from genetic diseases and our elderly to suffer from the likes of dementia and Alzheimer's.
Having already identified thousands of genetic disorders, we know what to look for when sequencing the embryos during IVF.
We will know, for example, if a child, if carried to term, will die of Tay-Sachs.
Data from European countries show that couples that receive a diagnosis of Down Syndrome from a pre-natal screening at three months, up to 93\% choose to abort \cite{Natoli2012}.
From this we can impute that almost no one will choose to implant an embryo that we know would be a child who will suffer and die from a genetic disorder.

The scary side of this is that we will make choices about our children based on factors beyond simple health.
We will know the probabilities of personality traits, intelligence, height etc. and will choose according to the cultural and economic norms of the day.
Take Sickle cell anemia as an example.
If you carry two alleles of the trait, you will likely die.
However, if you are a recessive carrier then you have much higher resistance against malaria at some cost to you hemoglobin's ability to carry oxygen.
This trait emerged as a desperate adaptation to the deadly threat of malaria.
However, malaria is nonexistent in the United States and sickle cell anemia is seen as a disease.
Through embryo selection, we can select out the gene for sickle cell anemia, leaving us vulnerable to a reemergence of malaria.
We have no idea what recessive traits we may be carrying that could help our species survive some threat that we have not faced in recorded history.

We do not know what trade-offs we have made through the course of our evolution.
To borrow from finance, we might be on a interior point of the efficient frontier.
We might be able to take two traits that are involved in a trade-off and optimize for both to reach a point of the efficient frontier where the trade-off actually starts to bind.
So far, we understand some the variables involved in our genetics but we do not understand most of them.
We used to say that there's a gene for that: the short gene, the tall gene, the smart gene etc.
We now have a poly-genic hypothesis that most of our traits are mediated by many genes, and often each gene has very little to say \cite{polygenic2016}.
To understand these tradeoffs, we will need a far greater understanding of genetics.

China's biophysicist He Jiankui is responsible for the birth of the first known genetically-edited babies in the world.
His "achievement" was initially lauded in China in the People's Daily as "a triumph of Chinese science".
(The article has since been removed from the online archive of the People's Daily.)
There was swift international condemnation of his actions \cite{crisprbabies2018}.
The event was controversial for a number of reasons.
The consent of the parents was misinformed.
He Jiankui did not obtain approval from the hospital.
The father had HIV while the mother did not.
In such a circumstance in a country with advanced medicine such as China, or the United States, there are many ways for the couple to have a child who will not inherit HIV.
He Jiankui targeted the CCR5 gene to make it similar to a mutation that some europeans have.
The mutation is two disrupted copies of the gene, granting the carrier increased resistance to HIV but making them more susceptible to the West Nile virus.
He Jiankui was not trying to solve an existing problem but was trying to make an enhancement.
A few months later, a report came out that mice with the same CCR5 mutation were faster at maze-running \cite{Regalado2019}.
This sparked rumors that the babies were engineered to be smarter.
Soon after, a study was released, analyzing data of almost half a million people from the UK Biobank, that found a link between people with the same CCR5 mutation and their lifespans \cite{Wei2019}.
People with the mutations were living shorter lives.
It is clear that our knowledge of genetics is limited and we do not know how even a single edited gene will affect our phenotypic traits.

The age of human genetic modification has begun.
The third Chinese genetically-engineered baby has probably already been born.
The Russian scientist Dennis Rebrikov has announced that he has five parents lined up.
This number is going to go from 2 to 3 to 8 and exponentially grow from there.
Within a decade, we will have thousands of genetically engineered babies.

\subsection{Differences in National Cultures}

National cultures around these technologies also differ.
We need to worry about the nations that are the most gung-ho about exploiting these technologies.
We need to worry about geo-political tensions with biology as the next battlefield where we will not even know if war has been declared.
China is the major concern.
Let us first visit the positive story from China.
China is building massive datasets by collecting information about its people.
While these datasets can and are being used to oppress people, they can also be used to do a lot of things that we consider good.
The datasets will enable their analysts to develop actionable insights about how cancers form, predisposition to certain diseases, and responses to certain treatments.
This is going to be an immense boon for Chinese healthcare and could also help us in the rest of the world.
China has many things.
They have a lot of money, incredibly talented people, a nearly unrestrained scientific culture, very few limits on what the government is prepared to do, and a national culture that aims to make the 21$^{st}$ century China's century.
These are translating into very aggressive applications of revolutionary science with biologists and data scientists at the forefront.

In the United States we have mane self-imposed, and in many cases rational, restrictions on which boundaries we are willing to push.
But is it prudent to hold back in this way if China is not going to be restrained?
In other words, does it become ethical to compete because lack of competition means ceding the game to an actor we think is less ethical?
We are having the same conversation around Lethal Autonomous Weapon Systems.
By developing these LAWS we lay the foundation for humans being wiped out, but by not developing them we empower the other country that is developing them.
While we need to be competitive on these technologies, we cannot lose our humanity in their pursuit.
This is a societal race and it is not that the country with the first genetically enhanced human wins the race.
The country that is first to figure out how to use the resources of its society as a whole to realize its objectives, whatever those may be, is the country that wins the race.

\subsection{Eugenics and Selection}

The word eugenics is used as a cudgel.
When someone starts talking of eugenics, the response is "Oh I'm against eugenics."
This is, in many ways, appropriate because so many terrible things have been done in the name of eugenics.
Perhaps we need a new word; something like "selection".
Yet when we talk of selection, we are immediately in Mengele territory, selecting who lives and who dies.
When you are in a situation in which there are some embryos in a dish and you have to pick one, what are the criteria by which you make the choice of which embryo to implant?
If you say that you want a child who is not going to die of some terrible genetic disease then that too is selection.
This is a normative choice set not in some abstract, objective world but in the context of our current reality.

With our choices set to change the future of human life, we have good reason to be terrified.
We have many people, and their children, who fled from Nazism to our country.
If we asked the Nazis what they were doing, they would have said that they were implementing Darwinism.
To make an extreme understatement, many jewish communities were on the losing end of a eugenics experiment writ large, gone mad.
And yet we are uncomfortable with the fact that eugenics has never been properly defined, and so does not really mean anything.
Mate selection is another form of genetic selection and if you decide that dinner-and-a-movie is eugenics then you have erred in drawing the line.
At the same time, we cannot have a free-for-all and let a hundred flowers bloom to see what the experiments produce.
It is not clear where good selection ends and bad selection begins.
In some way, we have been lying.
There is no way of drawing such a line.
In much the same way as in the abortion debate where neither the pro-choice not the pro-life camps make sense, neither the pro-hacking nor the anti-hacking positions make sense.
We are just left with a permanent struggle between spectra of choices.

It would be wrong for us to say that we will not employ genetic modification.
Who wants to have a kid who will die of some terrible genetic disease when we have the technology to save that kid?
Who wants to see their parents suffer from dementia if we can remove that condition as a possibility?
It would be insane for us to say that we have these powerful tools that will let us do unimaginable good but we are just not going to use them because these tools also have dangerous use-cases.
If that had been the case, we would not be using cars, we would not be using plows, or any other technology for that matter.
This is the position of the Amish: it's a slippery slope.
Some slippery slopes are great.
We solve one disease and learn from that how to solve another disease.
Dinner and a movie is a slippery slope to love and marriage.

We need to find a good place in the middle.
On the one hand He Jiankui is a villain and we have seen the danger of Nazi Germany.
On the other hand we are excited about the idea of freeing people from the risk of breast cancer and perhaps even enhancing cognitive ability.
However, is such sobriety even possible?
These technologies are becoming cheaper and more powerful.
This usually means that the technology is moving into someone's garage.
There is a growing do-it-yourself biology movement \cite{Kolodziejczyk2017}.
There was recently a Swedish man who created a functional nuclear reactor using the radioactive elements from hundreds of discarded smoke detectors \cite{Taylor2011}.
This cannot be stopped by simply criminalizing such ventures and having panels on ethics.
So far, the high cost of genetic modification, our lack of power and knowledge, and our clumsiness have allowed us to hide from these most dangerous questions.
With these factors disappearing, we will soon have to confront ourselves.

\subsection{Ethical Norms in Light of Genetic Differences}

In 1953 when the double-helix structure was elucidated by Watson and Crick and then in 1963 when the genetic code was figured out by Marshall Nirenberg, that group of scientists was shocked that we only pretended to care about our identity in the from of these letters.
We never really accepted what they found.
Every time the the scientists started talking about identity and characteristics, they found that we were so attached to our pre-genetic understanding of ourselves that we would not give it up.
We fight anyone tooth-and-nail who tries to tell us that our characteristics are really a consequence of information technology as developed by natural and sexual selection.
In many ways, we are stuck in a culture where we cannot update to assimilate the knowledge that we gained seventy years ago.

Our new-found ability to look under the hood is challenging many of the myths that we have developed over thousands of years.
We love the idea that you can be anything you want.
"The world is your oyster."
"Just set your heart to it kid and you'll go places."
This is just not true.
Take the narrow example of running.
Running is a standard human task that we all have evolved with.
We would run our prey to exhaustion in the hunter-gatherer era.
Yet there is nothing that most of us can do to become a champion marathon runner.
Between 1897, when we started keeping records of the Boston Marathon, and 1987 there were no winners of the Boston Marathon from either Kenya or Ethiopia.
Since 1987, while it has not exactly been total domination, a disproportionately large number of the champions hail from a very small group of people from the same valley in Kenya and Ethiopia.
They are from a single tribe, the Kelengin, of four million people, and even among them from a single subtribe subtribe, the Nandi, of one million people.
This is genetics.
While we all have a genetic range of possibility and we should each aspire to be on the top end of that range, whatever that means in the context of our potential, if I do not have the genetics to be the worlds fastest marathoner, or swimmer, or abstract mathematician, I am simply not going to get there.

In the example of marathon runners we went from a situation of wide diversity, where victory really was up for grabs and it was really about athleticism, to a situation in which we have found the special people that are wildly well adapted to radiate heat, which is the limiting factor in marathon running.
The world was not connected enough to bring in competitors from the Horn of Africa, and there was no prize for winning marathons.
With this infrastructure in place and the world becoming ever more connected, has there been a negative impact on the sport that we no longer even feel motivated to enter or watch?
In general, what happens when we have genetically optimized people in sports?
In many ways, this has already been happening and we just did not know it.
For example Michael Phelps, in addition to his extreme work ethic, training, and coaching, is genetically well-adapted to swimming \cite{Siebert2014}.
He is tall and slim with a long torso and short legs, which decrease viscous drag in water.
His arms are a few standard-deviations too long for a man his height and his palms and feet are so large so as to act as flippers.
Together, these traits give him greater ability to propel himself through water.
If we were to analyze our top performers in various sports, we would find that they too are genetically optimized for the tasks that they perform.

We have, rightfully so, a fear of living in a Plato's Republic type society but we also do not want to live in a society where the mission-critical functions are not being performed by those people that are most suited to handle those functions.
There is the old joke about the German chef and the Italian policeman.
Having the wrong person in the wrong job would hurt a society.
In the movie Gattaca, Ethan Hawke is a man who was born in the old-fashioned way.
He wants to get into the space program, pulls a lot of tricks, and in the end he succeeds.
We watched this movie and said "isn't it so great that that guy was so determined!"
Should Ethan Hawke have been arrested for his own safety because we do not want non-genetically enhanced people in the space program because, for example, they cannot survive the radiation of space?
If we find some person in a refugee camp who has the potential to be the next Mozart, should we ensure that we get resources to that person im order that they may enrich all our lives?
This idea that we will have predictive life is going to fundamentally change the stories that we tell ourselves.

Unfortunately, this clarity is hard to maintain when we move on to the next example.
Let us look at the number of female chess players among the best chess players in the world \cite{chess2020}.
We find that at the top level of chess only about one in a hundred are female.
It is only a single protein, SRY, that determines weather a proto-human-being becomes male or female.
We are not comfortable accepting the same story of genetic optimization for chess that we have for marathon running.
It might be that a similar conclusion comes out of our understanding of genetics about the ability to play chess the way that the Boston Marathon understanding comes out of the genetics.
However, I do not know what kind of a society I would be living in if I were comfortable saying that there are genetic disparities in cognitive ability.

Another dangerous example is that of Ashkenazi jews who represent one-quarter of one percent of the world's population and have won about 25 percent of the Nobel prizes in physics.
It is hard to think think of Judaism as also a breeding protocol of, over thousands of years, having the smartest people, the Rabbis, have as many children as possible, as opposed to the catholic priests who were supposed to be celibate.
It could also be that mathematical ability for lending money, which christians and muslims are not allowed allowed to do by religious doctrine, was selected for in jewish communities over thousands of years and only very recently did this ability become important in a world dominated by information technology and computer programming.
While we might accept that a small group of people from a geographically special region in Kenya and Ethiopia have a genetic advantage in marathon running, as soon as we kick it over into the realms of chess and physics, it does not feel good to even be thinking about these things.

These examples exist in the realm of our most taboo and difficult topics.
With the example of chess, it may be that the disparity is due to an obsessive trait rather than ability.
Perhaps if you want to be at the top of the chess pile, you have to be completely obsessed with chess.
Several studies have shown that men rate more strongly on obsessive traits than do women, though some studies disagree \cite{Mathis2011}.
It may also be that the rule-set of chess is such that men are more adapted to it.
We might easily see a game, call it chess-prime, with a different rule set for which women are better adapted.
It would not at all be surprising to see the upper echelons of chess-prime be dominated by women.
For example, speed-chess has a significantly different distribution of people at the highest level as compared to classic chess.
We have to ask ourselves weather we even want such things to be investigated.
With our current knowledge of genetics, we cannot make the claim that certain groups are just genetically well adapted to chess.

There is a cautionary tale to be told here.
Harvard's math department was almost entirely male up to the late 1980s.
They had a desire to see women do well but there was to good track record so they would admit one every year.
One year the one female student decided to defer so the next year there were two female students in the department.
They formed a support group and both did extremely well.
After this, there was a cohort that went through and now have successful careers.
There are still many unfair aspects left to mend, but the situation is improving \cite{Jackson2004}.
It would have been easy for us tell a soft tale of "Oh there are these genetic differences and we're looking at the tails of distributions ..."
While there is no way to hide from data, we cannot glibly assign an explanation when our knowledge of the real causes is so lacking.

In the end, there are no dispassionate arbiters.
We often pretend that we are objective and that we can make these conclusions.
And yet, relatively minor alterations to methodologies can yield wildly different results.
We need humility and modesty in our approach to what may seem, at first blush, to be extremely disturbing interpretations of the data.
We have to proceed in a scientific fashion and we cannot afford to always be thinking of the social consequences.
We also have to think of the social consequences and we cannot proceed blithely with the science.
