\section{Conclusion}
\label{sec:conclusion}

The genetic revolution has produced, and will continue to produce, fundamental changs in our society.
For one, our healthcare systems are going to evolve.
We will move away for the present systems of generalized healthcare towards personalized, precision healthcare.
Your doctor well need to know who you are.
This will require access to your health records, biometric information and, most of all, to your genetic information.
With the cost of genome sequencing trending towards zero, everyone is going to be sequenced as part of being in the healthcare system.
With this move towards precision medicine, there will be billions of people whose genetic and phenotypic information is going to be in massive big data pools.
We will use these datasets to increasingly demystify our biology and eventually move towards predictive medicine, healthcare, and life.

With the shift towards IVF, we will also change how we reproduce.
We will see an end to procreative sex.
Once we take conception outside of the human body, we will be able to apply the full range of our gene-editing toolkit in many incredible ways but also in ways that will scare a lot of people.
One fear is that we will make decisions about the future of our species based on what feel like eternal truths but are in fact transient fashions.
If we asked people today, they would say that they want a kid with good health, who would live a long life, be intelligent, and maybe tall.
These traits are fine to have and people who have them are thriving in our current environment.
However, our genetic diversity is not just some nice-to-have thing.
It is the sole survival strategy of any species.
We will have to consciously choose this diversity that we have inherited.
We will have to identify what we mean by diversity and we will have to celebrate it.
Otherwise, our species will court existential risks.

If only that were the full extent of the danger.
Individual actors now have the power to wreak havoc on a scale previously only possible for nations.
We are in the age of DIY bio-hacking.
Technologies like CRISPR/Cas9 will soon be cheap enough to edit genomes in a highschool biology project.
In 2017, a group of Canadian researchers used the tools of synthetic biology to create an active version of horsepox. as relative of smallpox \cite{Kupferschmidt2017}.
That had a grant of \$ 100,000 and little specialized knowledge of virology.
Today we could do this for \$ 50,000 and in five years, \$ 5,000.
With these bio-lego-bricks, we will have the tools to remake life.
Just as we have the tools to do incredible good, we also have the potential to do great evil.

We can moralize all we want but this going to play out, ironically, according to a system of selective pressures.
Selection is not only about biology.
Anything that has the three properties of variability, heritability, and differential success is going to behave in a darwinian fashion.
The social interpretation of this is that variability constitutes the human value of diversity, heritability has to do with what we call privilege, and differential success is inequality.
The maddening thing about biology is that we take this cherished value of diversity, we subject it to privilege, it produces inequality, and we take that inequality as the feedback into the system.
With bio-hacking we are going to move from obligate-heritability to facultative-heritability.
In effect we will become the designer in our own design.
We will break evolution and biology at a fundamental level.
After 3.8 billion years of evolution by random mutation and natural selection, we are about to turn a corner.
Will see a Cambrian explosion of successors to homo-sapien?
We cannot accurately imagine where this will go over the next centuries and millennia.
What we can say is that we have, over many thousands of years, developed ethical codes that help us live better lives.
At the very least, we need to fight to make sure that our best individual and collective values are integrated into the decision-making process.

"We are gods but for the wisdom." - Eric Weinstein.
